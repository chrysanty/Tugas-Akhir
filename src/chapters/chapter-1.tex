\chapter{Pendahuluan}

\section{Latar Belakang}
\label{sec:latarbelakang}

Dalam kehidupan sehari-hari, manusia sering kali menghadapi masalah pengambilan keputusan. Pengambilan keputusan tersebut akan berdampak pada keputusan berikutnya, dan akan terus berlanjut hingga mencapai suatu titik. Dalam ilmu matematika, terdapat teori permainan yang merupakan ilmu yang mempelajari pemodelan matematis untuk masalah dan kerja sama antara agen pengambil keputusan.

Teori permainan memiliki tiga aspek utama, yaitu: aksi, keluaran, dan preferensi. Tiga aspek ini saling mempengaruhi satu sama lain. Aksi akan mengakibatkan keluaran, keluaran akan memunculkan pilihan aksi selanjutnya, dan pilihan aksi diurutkan berdasarkan preferensi. Dalam kehidupan nyata, salah satu contoh teori permainan adalah kesetimbangan Nash (atau lebih dikenal sebagai \textit{Nash equilibrium}) yang menyatakan bahwa strategi optimal bagi seorang pemain adalah mengikuti strategi awal sambil mengetahui strategi lawan, dan mempertahankan strategi yang sama apabila seluruh pemain lain tidak mengubah strategi.

Permainan peran atau \textit{role-playing game} (RPG) adalah jenis permainan dimana pemain mengambil sebuah peran pada permainan dengan latar belakang fiksi. Pemain bertanggung jawab untuk memerankan suatu peran dan melakukan aksi. Jenis aksi yang dilakukan dapat bermacam-macam, seperti percakapan antar karakter atau melakukan suatu tindakan atau kegiatan. Aksi yang dilakukan merupakan suatu contoh pengambilan keputusan, dimana keputusan yang diambil akan berdampak pada pengembangan karakter selanjutnya. Pengembangan karakter dapat mempengaruhi akhir dari permainan.


Persona 4 Golden (selanjutnya akan disebut juga sebagai P4G) adalah permainan peran dari yang dibuat oleh perusahaan video gim Jepang, Atlus. Permainan P4G pertama kali rilis di PlayStation Vita pada tahun 2012 dan Windows pada tahun 2020. Permainan P4G memiliki latar waktu pada tahun 2011, dimana pemain utama (atau dikenal sebagai Narukami Yu) harus menyelesaikan misteri kasus pembunuhan berantai yang terjadi di kota Inaba.

Dalam permainan P4G, terdapat dua jenis aksi penting. Yang pertama, ketika pemain utama bersosialisasi dengan pemain lain. Dalam permainan ini, digambarkan bahwa setiap pemain memiliki \textbf{Persona}, atau atribut yang digunakan karakter ketika masuk ke \textit{dungeon}. Persona akan membantu karakter ketika bertarung dalam \textit{dungeon} melawan \textit{shadows}. Relasi antara pemain utama dengan karakter lain dinyatakan dalam bentuk \textit{social link}. Semakin erat relasi yang dimiliki antar karakter, maka \textit{social link} akan semakin tinggi. Artinya, Persona yang dimiliki akan semakin kuat dan \textit{dungeon} dapat diselesaikan dengan lebih cepat dan mudah. Yang kedua, ketika pemain utama masuk dalam \textit{dungeon}. Di dalam \textit{dungeon}, pemain utama akan menggunakan berbagai Persona untuk bertarung dan menyelamatkan pemain lain yang terjebak dalam \textit{dungeon} tersebut. Apabila pemain lain gagal diselamatkan, permainan berakhir dan permainan dinyatakan berakhir buruk.
Aksi yang dilakukan dalam \textit{dungeon} sangat krusial karena akan berdampak pada epilog permainan ini.

Rentang waktu permainan P4G terbatas dari bulan April 2011 hingga Maret 2012. Sehingga, pemain utama harus menstrategikan pemilihan keputusan terbaik yang dilakukan setiap hari dalam permainan. Tujuan utama dari tugas akhir ini adalah memaksimalkan \textit{social link} pemain utama dengan 23 Persona lainnya dalam rentang waktu tersebut. Sehingga, berdasarkan latar belakang yang sudah diuraikan, pada tugas akhir ini akan dimodelkan pemilihan keputusan terbaik untuk memaksimalkan \textit{social link} seluruh Persona. Model ini akan digunakan untuk memaksimalkan potensi penyelesaian permainan secara optimal, atau \textit{completion rate} 100\%.

\section{Rumusan Masalah}
Dengan waktu yang terbatas, pemain utama harus memaksimalkan \textit{social link} seluruh Persona. \textit{Social link} mempengaruhi keberhasilan penyelesaian permainan dan kemudahan pemain menyelesaikan permainan ketika berada di \textit{dungeon}.

Penelitian ini menjawab permasalahan pengambilan keputusan, yang secara spesifik dilakukan pada permainan Persona 4 Golden. Sehingga, rumusan masalah dari tugas akhir ini adalah sebagai berikut.
\begin{enumerate}
    \item Bagaimana cara mengkonstruksi fungsi \textit{payoff} sehingga pengambilan keputusan dalam permainan Persona 4 Golden optimal?
    \item Bagaimana algoritma penyelesaian masalah pemilihan keputusan akan diimplementasikan pada permainan Persona 4 Golden?
\end{enumerate}

\section{Tujuan}
Berdasarkan rumusan masalah yang telah ditulis pada subbab berikutnya, tujuan dari tugas akhir ini adalah sebagai berikut.
\begin{enumerate}
    \item Melakukan konstruksi fungsi \textit{payoff} untuk pengoptimalan pengambilan keputusan dalam permainan Persona 4 Golden.
    \item Mengimplementasikan algoritma masalah pengambilan keputusan pada permainan Persona 4 Golden.
\end{enumerate}

\section{Batasan Masalah}
Dalam tugas akhir ini, terdapat asumsi yang digunakan. Sehingga, batasan masalah yang ditetapkan untuk tugas akhir ini adalah sebagai berikut.
\begin{enumerate}
    \item Data yang digunakan berasal dari data hasil permainan, sehingga data yang tidak lengkap akan diasumsikan tidak ada atau berada pada suatu titik  atau batasan tertentu.
\end{enumerate}

\section{Metodologi}
Metodologi atau tahapan yang digunakan dalam tugas akhir ini adalah sebagai berikut.
\begin{enumerate}
    \item Mempelajari teori permainan dan permainan peran berdasarkan buku atau jurnal ilmiah yang relevan.
    \item Membuat algoritma masalah pengambilan keputusan pada permainan Persona 4 Golden.
    \item Melakukan implementasi dari algoritma masalah pengambilan keputusan pada permainan Persona 4 Golden.
    \item Melakukan analisis hasil implementasi algoritma masalah pengambilan keputusan pada permainan Persona 4 Golden.
\end{enumerate}
