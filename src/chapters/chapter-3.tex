\chapter{\textit{Social Link}}
\section{\textit{Social Link}}
\textit{Social link} merepresentasikan hubungan sosial antara karakter utama dengan karakter lainnya. Dalam permainan P4G, terdapat 23 \textit{social link} yang perlu ditingkatkan, dengan rincian:
\begin{enumerate}
    \item 3 \textit{social link} ditingkatkan secara otomatis.
    \item 2 \textit{social link} ditingkatkan dengan menyelesaikan misi.
    \item 1 \textit{social link} ditingkatkan secara manual sebanyak 6 level.
    \item 17 \textit{social link} ditingkatkan secara manual sebanyak 10 level.
\end{enumerate}
%Ceritain dari 23 social link ini, yang difokusin di TA ini ada 19 social link karena itu yang ngefek dari 2 alokasi waktunya

\section{Waktu dan Cuaca}
%https://megamitensei.fandom.com/wiki/Weather_Forecast
Dalam permainan P4G, karakter utama memiliki dua alokasi waktu untuk beraktivitas secara bebas, yaitu siang hari dan malam hari. Aktivitas yang dapat dilakukan dan \textit{social link} yang dapat ditingkatkan pada siang dan malam juga berbeda.
Namun, tidak semua hari di Inaba dapat digunakan untuk beraktivitas bebas, karena terdapat hari yang telah dijadwalkan oleh permainan untuk beraktivitas secara otomatis.
Beberapa contoh aktivitas otomatis yang mengakibatkan karakter tidak dapat melakukan aktivitas apapun, yaitu: ujian sekolah, acara sekolah, dan pencarian petunjuk kasus.

Cuaca sangat mempengaruhi aktivitas karakter utama selama tinggal di Inaba, karena pada hari dimana karakter tersebut dapat beraktivitas bebas, terdapat batasan aktivitas yang dapat dilakukan pada cuaca tertentu. Jenis-jenis cuaca pada permainan P4G adalah cuaca cerah, cuaca berawan, cuaca hujan, cuaca badai, cuaca berkabut, dan cuaca bersalju.

\subsection{Cuaca Cerah dan Berawan}
Pada cuaca cerah atau berawan, hampir seluruh aktivitas dapat dilakukan. Artinya, pada hari yang sesuai, apabila cuaca cerah atau berawan, karakter utama dapat meningkatkan \textit{social link} dengan karakter. Pada cuaca ini, karakter utama juga dapat bekerja sebagai pengasuh anak untuk mendapatkan gaji serta meningkatkan \textit{social link} dengan Eri Minami.

\subsection{Cuaca Hujan dan Badai}
%https://megamitensei.fandom.com/wiki/Weather_Forecast
Pada cuaca hujan atau badai, hampir seluruh \textit{social link} tidak dapat ditingkatkan. Akan tetapi, terdapat aktivitas-aktivitas yang mendapatkan \textit{bonus} atau hanya dapat dilakukan ketika dilakukan pada cuaca ini, seperti:
\begin{enumerate}
    \item Belajar, \textit{social qualities "Knowledge"} akan meningkat dua kali lipat.
    \item Makan \textit{Mega Beef Bowl} seharga 3.000 yen di \textit{The Chinese Diner Aiya} akan meningkatkan \textit{Social qualities "Understanding", "Knowledge", "Diligence} dan \textit{"Courage"}.
    \item Belanja di \textit{Shiroku Store} akan mendapatkan diskon 50\%.
    \item \textit{Shiroku Store Capsule Machine} dapat digunakan.
    \item \textit{Deidara Metalworks} akan tetap buka dan menawarkan diskon pada malam hujan.
    \item \textit{Super croquettes} dapat dibeli maksimal sebanyak lima di \textit{Souzai Daigaku}.
    \item Ikan yang lebih menantang lebih sering muncul di \textit{Samegawa Flood Plain} dan \textit{Shichiri Beach}.
    \item Kegiatan menangkap serangga di \textit{Tatsuhime Shrine} tidak dapat dilakukan.
    \item Beberapa \textit{shadows} di \textit{Midnight Channel} hanya muncul pada cuaca hujan.
    \item Peluang \textit{shadows} menghasilkan benda langka lebih besar.
    \item \textit{Skill rainy death} mengalami peningkatan \textit{critical rate} pada cuaca hujan.
    \item Hampir seluruh \textit{social link} tidak dapat ditingkatkan, kecuali Ayane Matsunaga/Yumi Ozawa, Fox, Nanako Dojima, Ryotaro Dojima, Sayoko Uehara, dan Shu Nakajima.
    \item Hampir seluruh misi \textit{outdoor} NPC tidak ada.
    \item Membaca buku di rumah karakter utama pada cuaca hujan dapat meningkatkan lebih banyak \textit{social qualities}.
\end{enumerate}

\subsection{Cuaca Berkabut}
Cuaca berkabut menggambarkan batas waktu menyelesaikan \textit{dungeon}. Ketika karakter utama memiliki misi menyelamatkan karakter lain dalam \textit{dungeon}, misi tersebut harus diselesaikan sebelum cuaca hujan tiga hari berturut-turut. Apabila \textit{dungeon} tidak berhasil diselesaikan sebelum cuaca berkabut, artinya permainan gagal diselesaikan.

\subsection{Cuaca Bersalju}
%https://megamitensei.fandom.com/wiki/Weather_Forecast
Cuaca bersalju hampir sama seperti cuaca cerah atau berawan. Namun, aktivitas seperti berkebun dan menangkap serangga tidak dapat dilakukan. Sebagai tambahan, karakter utama dapat mencari serangga dengan menggali salju di kebun Dojima. Perlu diperhatikan bahwa aktivitas-aktivitas diatas tidak akan meningkatkan \textit{social qualities} atau \textit{social link}, hanya akan menambah benda yang dimiliki karakter utama.


\section{\textit{Social Qualities}}
%https://www.rpgsite.net/feature/9822-persona-4-golden-part-time-jobs-every-job-requirements-rewards

\section{Perkembangan Kasus}



%Tujuan utama penulisan bab ini adalah untuk menguraikan rencana penyelesaian masalah tugas akhir yang akan dieksekusi secara utuh pada saat pelaksanaan Tugas Akhir II. Bab ini merupakan bab penutup Laporan Tugas Akhir I yang dapat dipandang sebagai bab yang akan menjembatani perpindahan ke proses pelaksanaan Tugas Akhir II. Pengembangan lebih lanjut dari bab ini dapat menjadi bagian dari bab Deskripsi Solusi pada Laporan Tugas Akhir.


\blindtext